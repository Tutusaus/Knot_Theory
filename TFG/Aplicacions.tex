% !TEX root = thesis.tex

\section{Aplicacions}\label{sec:Aplicacions}

\subsection{Computació amb nusos quàntics}\label{sec:nusosquantics}

Els ordinadors quàntics prometen realitzar càlculs que es creuen impossibles per als ordinadors convencionals. Alguns d’aquests càlculs tenen una gran importància pràctica. Pràcticament tots els mètodes d’encriptació utilitzats per a dades altament sensibles són vulnerables a un algoritme quàntic o altre.

El poder extra d’un ordinador quàntic prové del fet que opera sobre informació representada com a qubits, o bits quàntics, en lloc de bits. Un bit clàssic ordinari pot ser un 0 o un 1, i les arquitectures de microxip estàndard reforcen aquesta dicotomia de manera rigorosa. Un qubit, en canvi, pot estar en un estat anomenat superposició, que comporta proporcions de 0 i 1 coexistint alhora. Es pot pensar en els possibles estats d'un qubit son com els punts en una esfera. El pol nord és un 1 clàssic, el pol sud és un 0, i tots els punts intermedis són totes les superposicions possibles de 0 i 1. La llibertat que tenen els qubits per moure’s per tota l’esfera ajuda a donar als ordinadors quàntics les seves capacitats úniques.

Malauradament, sembla que els ordinadors quàntics són extremadament difícils de construir. Els qubits solen expressar-se com a certes propietats quàntiques de partícules atrapades, com ara ions atòmics individuals o electrons. Però els seus estats de superposició són extremadament fràgils i poden ser malmesos per les més mínimes interaccions fortuïtes amb l’entorn, que inclou tot el material que conforma l’ordinador mateix. 