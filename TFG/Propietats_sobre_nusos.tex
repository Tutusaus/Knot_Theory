% !TEX root = thesis.tex

\section{Propietats sobre nusos}\label{sec:Propietats_sobre_nusos}

En aquesta secció posarem de manifest diferents propietats d'un nus que més endavant tindran importància en la seva classificació.

\begin{definition}\label{def:ordre}
	Anomenem \underline{ordre} de $K$, $O(K)$ al mínim nombre de creuament que $K$ pot arribar a presentar donat un diagrama qualsevol.
\end{definition}

La Figura \ref{fig:culpritknot} demostra que l'ordre del nus Culprit és zero ja que aquest sempre és un nombre positiu. Tampoc és molt difícil veure que donat un nus $K$ qualsevol a aquest se li pot afegir creuaments mitjançant moviments RI, no obstant $O(K)$ es manté contant.

\begin{definition}\label{def:nusemmirallat}
	Diem que un nus és l'\underline{emmirallat} $\overline{K}$ d'un altre nus $K$ si aquest primer s'obté a partir de l'original canviant tots els creuaments de sobrepassos a sotapassos i a la inversa.
\end{definition}

La Figura \ref{fig:nus emmirallat} mostra un exemple d'un nus emmirallat.\\

\begin{figure}
	\resizebox{6.5cm}{!}{
		\begin{tikzpicture}[line width=0.5mm, use Hobby shortcut]
			\begin{scope}
				\begin{knot}[
					consider self intersections=true,
					%  draft mode=crossings,
					ignore endpoint intersections=false,
					flip crossing/.list={1,5,6},
					only when rendering/.style={
						%    show curve endpoints
					}
					]
					\strand ([closed]0,0) .. (1.5,1) .. (.5,2) .. (-.5,1) .. (.5,0) .. (0,-.5) .. (-.5,0) .. (.5,1) .. (-.5,2) .. (-1.5,1) .. (0,0);
				\end{knot}
				\path (0,-.7);
			\end{scope}
			
			\begin{scope}[xshift=-5cm]
				\begin{knot}[
					consider self intersections=true,
					%  draft mode=crossings,
					ignore endpoint intersections=false,
					flip crossing/.list={3,4},
					only when rendering/.style={
						%    show curve endpoints
					}
					]
					\strand ([closed]0,0) .. (1.5,1) .. (.5,2) .. (-.5,1) .. (.5,0) .. (0,-.5) .. (-.5,0) .. (.5,1) .. (-.5,2) .. (-1.5,1) .. (0,0);
				\end{knot}
				\path (0,-.7);
			\end{scope}
		\end{tikzpicture}
	}
	\caption{A l'esquerra el nus $4_1$ o Figure Eight en Anglès. A la dreta, el seu nus emmirallat.}\label{fig:nus emmirallat}
\end{figure}

D'aquesta manera, si un nus $K$ és equivalent a $\overline{K}$, llavors diem que és \textit{amfiquiral}, sinó diem que és \textit{quiral}. El nus $4_1$ de la Figura \ref{fig:nus emmirallat} és un exemple de nus amfiquiral, però no sempre és veritat que un nus $K$ sigui equivalent a $\overline{K}$ com veurem més endavant.

\begin{definition}\label{def:nusinvers}
	Diem que un nus és l'\underline{invers} $rK$ d'un altre nus $K$ si aquest primer s'obté a partir de l'original canviant-ne l'orientació.
\end{definition}

És clar que un nus pot ser orientat de dues maneres diferents, escollir una d'aquestes orientacions és informació extra sobre el nus que pot o no ser donada. De manera similar a la Definició \ref{def:nusemmirallat}, no sempre un nus és equivalent al  seu invers.

\begin{figure}
	\centering
	\includegraphics[width=0.6\linewidth]{img/orientació.png}
	\caption{Les dues úniques orientacions del nus trivial. És evident que aquests dos diagrames son equivalents.}\label{fig:nusorientat}
\end{figure}

\begin{definition}
	Diem que un nus $K$ és \underline{primer} si no és el nus trivial i si $K=K_1+K_2$ implica que $K_1$ o $K_2$ és el nus trivial.
\end{definition}

En aquest cas, el símbol $+$ fa referència a l'operació suma. Donats dos nusos orientats $K$ i $K'$, aquests poden ser sumats posant-los un al costat de l'altre i unint-los de manera que es preservi l'orientació. Aquesta operació està ben definida sota equivalència. A més, és commutativa, associativa, té element neutre i es comporta de forma similar al producte de nombres enters. De fet, caldria pensar en la suma de nusos com la suma conexa de varietats desenvolupada en un curs fonamental de topologia.\\

\begin{figure}
	\centering
	\includegraphics[width=0.9\linewidth]{img/nussuma.png}
	\caption{Suma de dos nusos qualssevol.}\label{fig:nussuma}
\end{figure}

La Figura \ref{fig:knotable} és una taula que conté tots els nusos primers de fins a 9 creuaments. Cadascun rep un nom conformat per un parell de nombres; el primer de tots correspon al seu ordre i el segon és un subíndex històricament assignat a aquell nus en concret. Les primeres tabulacions que es coneixen van fer-se per Tait l'any 1860 mentre aquest estudiava l'àtom. Aquesta taula negligeix el fet que possiblement un mateix nus $K$ sigui diferent a $\overline{K}$, $rK$ o $\overline{rK}$. D'aquesta manera, cada nus de la taula correspon a un, dos o quatre nusos de $S^3$ mitjançant les operacions definides a \ref{def:nusemmirallat} i \ref{def:nusinvers}.

\begin{figure}
	\centering
	\includegraphics[width=0.9\linewidth]{img/knottable.png}
	\caption{Taula de nusos fins a ordre 9.}\label{fig:knotable}
\end{figure}

\begin{definition}\label{def:nusalternat}
	Diem que $K$ és un nus \underline{alternat} si a mesura que anem recorrent el nus, els creuaments alternen de sobrepassos a sotapassos.
\end{definition}

El nus $4_1$ n'és un exemple. De fet, cal anar fins a $8_{19}$ per trobar el primer nus no alternat i doncs això és una conseqüència del baix ordre dels nusos. Es pot veure que donat un ordre qualsevol, sempre existeix com a mínim un nus alternat. Ara bé, aquests disminueixen exponencialment a mesura que incrementem l'ordre del nus. Aquesta classe de nusos tenen bones propietats.